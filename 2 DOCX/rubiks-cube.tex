\documentclass{article}
\usepackage{amsmath}
\usepackage{graphicx}
\usepackage{listings}
\usepackage{xcolor}

\title{Rubik's Cube Visualization and Interaction in C++ using OpenGL}
\author{Arfaz Hussain}
\date{\today}

\begin{document}

\maketitle

\section{Introduction}
This document provides an in-depth explanation of a C++ program designed to visualize and interact with a Rubik's Cube using OpenGL. The code demonstrates advanced 3D graphics programming concepts, including rotations, transformations, and event handling for keyboard and mouse input. The program allows users to interact with a simulated Rubik's Cube, manipulating its layers using various keyboard commands.

\section{Dependencies and Libraries}
The program utilizes the following libraries:
\begin{itemize}
    \item \texttt{<GL/gl.h>} - The core OpenGL library for rendering.
    \item \texttt{<GL/glut.h>} - The GLUT library for handling windowing, input, and event processing.
    \item \texttt{<vector>} - The C++ Standard Library for dynamic array handling.
\end{itemize}

\section{Code Structure}
The code is structured into several functions, each responsible for specific aspects of visualization and interaction:
\begin{itemize}
    \item Global variable declarations and structures.
    \item Functions for loading visualization parameters and setting up the camera.
    \item Functions for applying rotations and drawing individual cubes and the entire Rubik's Cube.
    \item Functions for handling user input via keyboard and mouse.
    \item Main function to initialize the OpenGL context and enter the event loop.
\end{itemize}

\section{Function Descriptions}

\subsection{\texttt{apply\_rotation(GLfloat angle)}}
This function is responsible for applying a rotation to a specific face of the Rubik's Cube. It first identifies which face (X, Y, or Z) is selected based on global variables \texttt{x0}, \texttt{xK}, \texttt{y0}, \texttt{yK}, \texttt{z0}, \texttt{zK}. The function uses these indices to modify the \texttt{cube\_rotations} array that keeps track of all the rotations applied to each cube.

\subsection{\texttt{reset\_selected\_face()}}
Resets the selected face for rotation to the default entire cube selection.

\subsection{\texttt{errand(int x, int y, int z)}}
This function is an example of memory management. It creates a temporary integer pointer, allocates memory, and then deallocates it. This could serve as an example or a placeholder for error handling or other computational tasks.

\subsection{\texttt{set\_camera()}}
Sets up the camera view using \texttt{gluLookAt()} to provide a perspective for rendering the 3D cube.

\subsection{\texttt{draw\_cube(int x, int y, int z)}}
Responsible for drawing an individual cube in the Rubik's Cube grid. It applies all stored rotations and translates the cube to its correct position in the 3D space.

\subsection{\texttt{draw\_func()}}
The core display function that renders the entire Rubik's Cube. It clears the screen, resets transformations, applies camera settings, and draws all 27 sub-cubes in their respective positions.

\subsection{\texttt{init\_func()}}
Initializes OpenGL settings such as lighting, shading, and material properties. This function sets up the environment for 3D rendering.

\subsection{\texttt{load\_visualization\_parameters()}}
Loads perspective projection parameters to define how the 3D scene is projected onto the 2D screen.

\subsection{\texttt{reshape\_func(GLsizei w, GLsizei h)}}
Handles window resizing events to adjust the aspect ratio and projection settings.

\subsection{\texttt{keyboard\_func(unsigned char key, int x, int y)}}
Handles keyboard input for rotating the entire Rubik's Cube or its individual faces. It maps various keys to specific transformations and updates.

\subsection{\texttt{mouse\_func(int b, int s, int x, int y)}}
Handles mouse input for zooming in and out of the 3D scene by adjusting the viewing angle.

\subsection{\texttt{main(int argc, char **argv)}}
The entry point of the program. It initializes GLUT, sets up the display mode and window size, registers callback functions, and starts the GLUT event processing loop.

\section{Compilation and Execution Instructions}
To compile and execute the program, follow these steps:
\begin{enumerate}
    \item Ensure you have OpenGL and GLUT installed on your system.
    \item Use the following command to compile the program:
    \begin{verbatim}
    g++ -o rubiks_cube rubiks_cube.cpp -lGL -lGLU -lglut
    \end{verbatim}
    \item Run the program using texttt{make} or using the following command:
    \begin{verbatim}
    ./rubiks_cube
    \end{verbatim}
    \item Use keyboard and mouse inputs (yet to be implemented) to interact with the Rubik's Cube.
\end{enumerate}

\section{Source Code}
The complete source code for this Rubik's Cube visualization can be found on GitHub at: \texttt{https://github.com/arfazhxss/rubiks-cube-cpp}.

\end{document}

